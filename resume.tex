% LaTeX resume using res.cls
\documentclass[margin]{res}
%\usepackage{helvetica} % uses helvetica postscript font (download helvetica.sty)
%\usepackage{newcent}   % uses new century schoolbook postscript font 
\setlength{\textwidth}{5.1in} % set width of text portion

\begin{document}

% Center the name over the entire width of resume:
 \moveleft.5\hoffset\centerline{\large\bf Michael Hunter}
% Draw a horizontal line the whole width of resume:
 \moveleft\hoffset\vbox{\hrule width\resumewidth height 1pt}\smallskip
% address begins here
% Again, the address lines must be centered over entire width of resume:
 \moveleft.5\hoffset\centerline{225 Fell St. Apt. 9}
 \moveleft.5\hoffset\centerline{San Francisco, CA 94102}
 \moveleft.5\hoffset\centerline{tahoemph@gmail.com}
 \moveleft.5\hoffset\centerline{775-298-6864}


\begin{resume}
 
\section{Education} {\sl MS Mathematics } CSULB, June 1994

                    {\sl AB Mathematics } Occidental College, June 1989
 
\section{Employment} {\sl Principal Engineer} iMatchative \hfill January 2015 - Present

                Leading the migration of a financial analytics website onto a modern stack
                in order to increase feature velocity.
 
                {\sl Software Architect} Say Media \hfill September 2012 - January 2015

                Initially my role at Say Media was as a Senior Software Engineer working
                on an internal sales applications written using Django. I learned the
                codebase, learned the business, and helped those tools grow with the
                business.

                The business was transitioning away from additional internal tooling
                to building a custom CMS to support our own sites. Later that grew to
                supporting an expanding set of sites we brought onto the platform. In
                the broad sense I was part of a small group of architects responsible
                for choosing and applying technology. My specific focus was on the
                backend services (Python, MySQL) and on the CMS admin (composing and
                other editorial functions in AngularJS). We built an editor using
                content editable which allowed the user to build a representation of
                the document. We then convert that into a renderable representation
                based on the target environment. The user had the ability to do normal
                editorial operations such as applying various types of styling and
                putting various embeds into the document.

                A peer and I designed the concepts behind grouping and representation of
                non-article data when faced with a site with a large amount of non-article
                data (biography.com). The core concept was a type with base properties
                allowing us to display a basic representation on the site. We are able to
                (and have) extended that type to support other uses including our basic
                grouping operation. I helped design an editor around introspection of that
                type which allows us to quickly bring up the ability to build content.

                Some of my coding responsibilities during this time included bringing
                up the initial site migration system and migrating several sites,
                implementing the new custom type on the back end, extending how we
                manage cache invalidation, prototyping parts of the introspected editor,
                root causing many of our deeper problem, and architecting and helping
                design the majority of the backend and admin software.

                My current interests are in how we adjust the structure of the
                editor so it is easier to grow. Some of this is around applying
                lessons we have learned building a large single page application with
                AngularJS. Additionally I am working on increasing the performance of
                the backend in support of better UI responsiveness.

                {\sl Senior Data Engineer} Zephyr Associates \hfill January 2011 - September 2012

                Developed an ETL system to replace an existing data
                pipeline built on DTS using Python 2.7 and SQL Server. The
                pipeline is used to process roughly 60 data feeds into
                more than 100 databases which are shipped to Zephyr's
                customers. I've reduced the time from getting a data feed
                to shipping a database by a factor of 2. Am currently
                working on building a production system so that we can
                further increase our rate and reliability of database
                production.

                Helped replace an aging database install mechanism with
                one built on C\# and Install Shield.

                Helped replace an older ISAM database format with SQLite.

                At the end of my tenure Zephyr's database schema was
                denormalized and hard to grow. My research included
                mechanisms to be used for reducing the amount of
                curating we need to do and mechanisms to be used for
                scaling our databases in size without breaking backwards
                compatibility.

                {\sl Staff Engineer} Sun Microsystems, Inc (Oracle) \hfill June 2002 - July 2010

                Updated the IPv6 basic API (RFC 3493) and implemented the IPv6 advanced API (RFC 3542).

                Worked as an integral part of the IPv6 team working
                to deliver significant upgrades and bug fixes for the
                Solaris 10 release.

                Worked on NWAM (Network Automagic) from conception
                through several releases. The focus of NWAM is to
                simplify and automate network configuration. For many
                cases software should be able to figure out what the
                user wants to do. For many others the set of choices are
                fairly small and the user shouldn't have to dig through
                uninteresting options. By combining a set of daemons
                which implement policy and a UI which provides the user
                with the ability to control that policy NWAM is able
                to significantly simplify network configuration. The
                team struggled to find the right balance between the
                common case (a laptop) and wanting to be able to specify
                more complicated situations (for example, when the user
                encounters a certain network NWAM should automatically
                create a tunnel to be used for communication). I
                provided an early prototype which later grew into an
                initial delivery vehicle. Later I worked on providing
                components for a more complete implementation.

                {\sl Senior Software Engineer} Extreme Networks \hfill December 2000 - April 2002

                Implemented a memory allocator and buffer management library for a MIPS
                R4000 operating a T3 line card. I used a fixed buffer layout choosing
                cacheability based upon memory type in order to be able to reach line
                rate.

                Ported the Telenetworks Frame Relay stack to a T1 blade. We were mixing
                and matching versions of the Telenetworks Frame Relay stack and the
                Telenetworks PPP stack running under a custom executive. I used a buffer
                scheme involving variant types of headers and lazily switched between
                them under application control in order to get differing buffer schemes
                to work together.

                Added Cisco HDLC support to a T1 blade.

                Implemented EDP (PPP Control Protocol)

                {\sl Senior Software Engineer} Redback  Networks \hfill August 2000 - December 2000

                Implemented Cisco HDLC for a pre-release optical line card
                running on custom processors. These processors provided
                fast access to packet memory but very slow access to the
                rest of memory. All coding had to be done in assembly.

                Designed and implemented an infrastructure to request
                information from a line card. This included a framework to
                help the user (developer) implement the CLI and marshall
                messages on the host processor and a framework on the data
                processor to parse the requests and return responses.

                {\sl Technical Lead } Cisco System \hfill February 1999 - August 2000

                Delivered Cisco HDLC to a pre-release system aimed at environments with
                high reliability and scalability needs. Cisco HDLC was used to provide
                interoperability with existing platforms. It is a very simple protocol
                that was used to test most system features such as restartability.


                Provided design input, implementation, and debugging support for
                interface management in the same environment. Interface management was
                complicated by the need to manage multiple types of interfaces across
                multiple processors connected by a variety of media from fairly slow to
                extremely fast.

                Designed and implemented BACP and parts of VPDN for the same
                environment. This was made difficult by needing to support both local
                and non-local interfaces.

                {\sl Software Design Engineer } QNX Software System Ltd.\hfill June 1995 - August 1998 \\
                {\sl Manager TCP/IP Technologies } \hfill Auguest 1998 - February 1999

                Brought a port of the BSD 4.3 Reno TCP/IP stack to
                market. Responsible for the TCP/IP stack, utilities,
                documentation, and support.

                Wrote a TCP/IP stack for a new version of the QNX
                OS (Neutrino). This stack supports a single network
                interface, IP, ICMP, UDP, and TCP. My stack met its
                code budget of 40K of x86 code. Later added multiple
                I/F support.

                Ported the BSD 4.4 stack to QNX 4. Since QNX 4 is a
                microkernel architecture this involved wrapping the
                BSD 4.4 stack in an emulation of its native kernel
                environment.

                Wrote various utilities for Neutrino including a general
                system interrogation utility (“super” ps) and a
                filesystem manager to support unix pipes.

                Select telecom stack vendor and guided port.

                Responsible for TCP/IP technology development from choice
                of which technologies to pursue to detailed design and
                scheduling of coding tasks.

                Wrote HTML to Troff (man macros, tbl) converter.

                {\sl Senior Software Engineer } Air Touch Teletrac\hfill January 1993 - May 1995

                Developed a distributed real time simulation under QNX
                2.x in support of white box testing a RF switch.

                Led the development of a transition plan from QNX 2.x
                to QNX 4.x.

                Evaluated and test communication and computing hardware
                for a multiprocessor “small footprint” system.

                Ported a complex communication simulation from Quick
                BASIC 4.5 to Visual BASIC 3.0 under very tight schedule.

                Participated in the full development of an autonomous RF
                switch to replace operator intensive one. This switch
                will be hosted on a STD32 backplane w/80[45]86 class
                processors running QNX 4.x. Wrote RF scheduling, generic
                state transition, and telecommunications code (TCP/IP,
                async). Developed lib code for message handling and
                error logging in c for team. Many of my own processes
                are written in C++.

                Build a QNX 4 IO Manager that allowed for
                “redirectable” and multiplexed I/O to/from a variety
                of sources.

                Worked as QNX 4 Guru and design expert in a team
                environment.

                {\sl Software Desin Engineer } Microcosm, Inc. \hfill June 1987 - January 1993

                Developed requirements document, helped design, and wrote
                parts of Space View, a spacecraft geometry visualization
                tool. Initial version being written on Sun SPARCstation
                using the OpenWindows GUI in C++. Designed and implemented
                an interprogram communication scheme utilizing TCP/IP
                to exchange data with a related product written by a
                different company.

                Designed, coded, and tested real time flight software for
                an autonomous spacecraft. Wrote math utilities, system
                executive, and Mil-Std-1553b output handler. Integrated
                software on a Mil-Std-1750a target and 1553b bus.

                Designed, documented, and implemented real time, flight
                hardware in the loop, simulation engine in Mil-Std-1815a
                (Ada) under VMS. Wrote orbit propagator to provide a
                reference trajectory for spacecraft simulation. Presented
                aspects of the simulation to customer at design reviews.

                Co-author of a winner Phase I SBIR proposal applying a
                rule based system to space attitude determination and
                telemetry data editing.

                Lead programmer of a spherical geometry visualization
                tool implemented on an IBM PC in standard c under
                MS-DOS. Worked with Hercules/EGA/VGA graphics and OS
                interfaces.

\end{resume}
\end{document}




